\documentclass{article}
\usepackage{fontspec, fullpage}
\usepackage{polyglossia}
\usepackage{amsmath, amssymb, bbm, amsthm}
\setmainlanguage{hebrew}
\setmainfont{Times New Roman}
% \newfontfamily{\hebrewfont}{New Peninim MT}
\begin{document}
\title{תרגול 3 חשבון אינפיניטסימלי 1 שנת 2021/2}
\author{ישראל הבר}
\maketitle

\newtheorem{theorem}{משפט}
\newtheorem{exercise}{תרגיל}
\newtheorem{homeexercise}{תרגיל לבית}
\newtheorem{example}{דוגמה}
\theoremstyle{definition}
\newtheorem{definition}{הגדרה}
\newtheorem{notation}{סימון}
\newtheorem{claim}{טענה}
\renewcommand\qedsymbol{$\blacksquare$}
\newcommand{\limtoinfty}{\underset{n\rightarrow\infty}{\lim}}

\begin{exercise}
יהיו 
$\alpha,\beta>0$
ויהיו סדרות כך ש 
\[\begin{cases} a_1=\alpha,\quad b_1=\beta \\ a_{n+1}=\frac{a_n+b_n}{2} \\ b_{n+1}=\sqrt{a_n b_n} \end{cases}\]
הוכיחו כי הסדרות הנ"ל מתכנסות.
\end{exercise}
\begin{proof}
נשים לב כי 
\[a_n-b_n = \frac{a_{n-1}+b_{n-1}}{2} - \sqrt{a_{n-1}b_{n-1}} = \frac{a_{n-1}-2\sqrt{a_{n-1}b_{n-1}}+b_{n-1}}{2}=\frac{\left(\sqrt{a_{n-1}}-\sqrt{b_{n-1}}\right)^2}{2}\geq 0\]
ולכן עבור 
$n\geq 2$
מתקיים 
$a_n\geq b_n$.
נוכיח באינדוקציה כי עבור 
$n\geq 2$
מתקיים גם כי 
$a_n$
מונוטונית יורדת ו
$b_n$
מונוטונית עולה. וזאת מכיוון ש -
\begin{align*}
	a_{n+1} &= \frac{a_n+b_n}{2}\leq \frac{a_n+a_n}{2} = a_n \\
	b_{n+1} &= \sqrt{a_n b_n} \geq \sqrt{b_n\cdot b_n} = b_n
\end{align*}
ולכן סך הכל לכל 
$n\geq 2$
מתקיים כי
\[b_2\leq b_n\leq a_n\leq a_2\]
סך הכל 
\[\{a_n\}_{n=2}^{\infty},\quad \{b_n\}_{n=2}^{\infty}\]
סדרות חסומות ומונוטוניות נקבל כי הן גם מתכנסות. שתיהן זנב של הסדרות המקוריות שלנו ולכן גם הן מתכנסות.
\end{proof}

\begin{exercise}
יהי 
$0<c<1$
ונגדיר סדרה - 
\[\begin{cases}&a_1 = c\\ &a_{n+1}=\frac{1}{2}c+\frac{1}{2}a_n^2\end{cases}\]
הוכיחו כי הסדרה מתכנסת ומצאו את הגבול.
\end{exercise}
\begin{proof}
נוכיח באינדוקציה כי 
$a_{n+1}-a_n<0$. \\\\
\textbf{בסיס:}
\[a_2-a_1=\frac{c}{2}+\frac{c^2}{2}-c = \frac{c^2-c}{2}<0 \quad(0<c^2<c)\]
נשים לב גם שכל איבר בסדרה חיובי וזאת מכיוון שבהגדת הסדרה מעורבים רק מספרים חיוביים וסכומים  של מספרים חיוביים. \\\\
\textbf{צעד:} 
\[a_{n+1}-a_n = \frac{1}{2} c+\frac{1}{2}a_n^2 - \left(\frac{1}{2}c+\frac{1}{2} a_{n-1}^2\right) = \frac{1}{2}\left(a_n^2-a_{n-1}^2\right)=\frac{(a_n-a_{n-1})(a_n+a_{n-1})}{2}<0\]
\end{proof}
 ולכן הסדרה מונוטונית יורדת. בנוסף הסדרה חסומה בין 0 ו1. ולכן הסדרה מתכנסת. עכשיו נמצא את הגבול. נניח כי הגבול הוא L. נשים לב כי מתקיים ש 
\[\underset{n\rightarrow\infty}{\lim} a_{n+1} = \underset{n\rightarrow\infty}{\lim}a_n(=L)\]
ולכן 
\[L=\underset{n\rightarrow\infty}{\lim} a_{n+1} = \underset{n\rightarrow\infty}{\lim}\left(\frac{1}{2}c+\frac{1}{2}a_n^2\right) = \frac{1}{2}\left(c+L^2\right) \]
ולכן 
\[L^2-2L+c=0\]
כלומר 
\[L_{1,2}=\frac{2\pm \sqrt{4-4c}}{2} = 1\pm \sqrt{1-c}\]
נשים לב כי L לא יכולה להיות גדולה מ1 ולכן 
\[L=1-\sqrt{1-c}\]
\begin{exercise}
הוכיחו כי הסדרה הבאה מתכנסת ומצאות את גבולה - 
\[\begin{cases}&a_1= \sqrt{2} \\ &a_{n+1}= \sqrt{2+a_n}\end{cases}\]
\end{exercise}
\begin{proof}
קודם כל נוכיח כי 
\[\forall n\in\mathbb{N}: 0\leq a_n\leq 2\]
את זה אפשר לעשות באינדוקציה - \\\\
\textbf{בסיס:}
\[0\leq\sqrt{2}(=a_1)\leq 2\]
\textbf{צעד:}
\[0\leq a_{n+1}=\sqrt{a_n+2}\leq \sqrt{2+2} =2\]
בנוסף נשים לב כי
\[a_{n+1}-a_n = \sqrt{2+\sqrt{a_n}}-a_n = \frac{2+a_n-a_n^2}{\sqrt{2+\sqrt{a_n}} +a_n}\]
לכן 
$a_{n+1}\geq a_n$
אם ורק אם 
\begin{align*}
2+a_n-a_n^2&\geq 0 \Longleftrightarrow\\
a_n^2-a_n-2&\leq 0 \Longleftrightarrow \\
-1\leq a_n\leq 2
\end{align*}
ולכן לכל 
$n\in\mathbb{N}$ 
מתקיים 
$a_{n+1}\geq a_n$
כלומר הסדרה מונוטונית. הוכחנו כאמור שהיא גם חסומה ולכן היא מתכנסת. בדומה לתרגיל שעבר נחשב את הגבול. נניח כי הגבול הוא L - 
\[L = \limtoinfty a_{n+1} = \limtoinfty \sqrt{2+a_n} = \sqrt{2+L}\]
ולכן מתקיים כי 
\begin{align*}
L^2&=2+L \\
L^2&-L-2=0 \\
(L-&2)(L+1) = 0
\end{align*}
כמובן 
$L\neq -1$
ולכן 
$L=2$.
\end{proof}
\begin{exercise}
תהי הסדרה הבאה - 
\[a_n:=\frac{\sum_{i=1}^k \alpha_in^i}{\sum_{i=1}^k \beta_in^i}\]
כאשר מניחים שהיא מוגדרת תמיד ומתקיים 
$\alpha_n,\beta_n\neq 0$
מצאו את גבול הסדרה.
\end{exercise}
\begin{proof}
זה יהיה תוצאה של אריתמטיקה של גבולו, נשים לי כי ניתן לכתוב את הסדרה גם באופן הבא - 
\[a_n = \frac{\sum_{i=1}^k \alpha_in^{i-k}}{\sum_{i=1}^k \beta_in^{i-k}}\]
נשים לב שהסדרות במונה ובמכנה מתכנסות וזאת מכיוון שלכל 
$i<k$
מתקיים כי 
\[\alpha_i n^{i-k}\underset{n\rightarrow{\infty}}{\longrightarrow} 0 \quad \beta_i n^{i-k}\underset{n\rightarrow{\infty}}{\longrightarrow} 0\]
ועבור 
$i=k$
מתקיים 
\[\alpha_i n^{i-k} = \alpha_i\cdot 1\underset{n\rightarrow{\infty}}{\longrightarrow} \alpha_k \quad \beta_i n^{i-k} = \beta_i \cdot 1\underset{n\rightarrow{\infty}}{\longrightarrow} \beta_k\]
ולכן 2 הסדרות מתכנסות והמכנה לא שואף לאפס ומאריתמתיקת גבולות נקבל - 
\[\limtoinfty a_n = \frac{\alpha_k}{\beta_k}\]
\end{proof}
\begin{example}
למשל עבור - 
\[a_n = \frac{3n^7+5n^2+1}{6n^7+n^4}\]
נקבל כי 
\[\limtoinfty a_n = \limtoinfty \frac{3+\frac{5}{n^5} + \frac{1}{n^7}}{6+\frac{1}{n^3} = \frac{3}{6}} = \frac{3}{6}=\frac{1}{2}\]
\end{example}
\begin{exercise}
הוכח כי הסדרה הבאה מתכנסת ומצא את גבולה 
\[\begin{cases}&a_1=\frac{3}{4} \\ &a_{n+1} = \frac{a_n+2}{a_n+3}\end{cases}\]
\end{exercise}
\begin{proof}
באינדוקציה ניתן להוכיח כי 
\[\forall n\in\mathbb{N}:\quad 0<a_n<1\]
בנוסף לכך ניתן להוכיח באינדוקציה כי הסדרה מונוטונית יורדת - \\\\
\textbf{בסיס:}
\[a_2 = \frac{11}{15}<\frac{44}{60}<\frac{45}{60} = \frac{3}{4}=a_1\]
\textbf{צעד:}
\[a_{n+2} = \frac{a_{n+1}+2}{a_{n+1}+3} = 1-\frac{1}{a_{n+1}+3}<1-\frac{1}{a_n+3}=a_{n+1}\]
סך הכל הסדרה מונוטונית יורדת וחסומה. לכן הסדרה גם מתכנסת. עכשיו נמצא את הגבול. 
\[L=\limtoinfty a_{n+1} = \limtoinfty \frac{a_n+2}{a_n+3} = \frac{L+2}{L+3}\]
ולכן נקבל - 
\begin{align*}
	L&=\frac{L+2}{L+3} \\
	L^2+3L &= L+2 \\
	L^2 + 2L - 2 &= 0 \\
	L_{1,2}&=\frac{-2\pm \sqrt{2^2+2\cdot 4}}{2} = -1\pm\sqrt{3}
\end{align*}
כמובן L אי שלילי ולכן נקבל כי
\[L=\sqrt{3}-1\approx 0.73205\]
\end{proof}
\end{document}