\documentclass{article}
\usepackage{fontspec}
\usepackage{polyglossia}
\usepackage{amsmath, amssymb, bbm, amsthm, graphicx, subfiles}
\setmainlanguage{hebrew}
\setmainfont{Times New Roman}
% \newfontfamily{\hebrewfont}{New Peninim MT}
\begin{document}
\title{עברית בלאטך}
\author{ישראל}
\maketitle

זה דוגמה למסמך בעברית. אופס הנה משוואה בתוך השורה
$y=x^2$
רגע אנחנו עדיין בתוך השורה? כן אני יודע זה הזוי אבל צריך להתרגל לזה. רגע רגע רגע מה אפשר לשים משוואה רגילה גם בשורה מעצמה?
\[\int_0^{\infty} e^{-t} dt = 1\]
אני לא מאמין איזה מדהים זה, יש אנשים שרוצים להשתמש בליך ולא בלאטך? איזה הזויים. אם רק היה אבל אפשר לכתוב במסמך עם כמה שפות בצורה קלה זה היה פשוט מדהים ויוצא דופן.

\end{document}
