%\title{Hebrew document in WriteLatex - מסמך בעברית}
\documentclass{article}
\usepackage[utf8x]{inputenc}
\usepackage[english,hebrew]{babel}
\selectlanguage{hebrew}
\usepackage[top=2cm,bottom=2cm,left=2.5cm,right=2cm]{geometry}

\usepackage{amsfonts,amsmath}

\title{תשימו שם של קורס}
\author{שם הסטודנט:
%%%%%% WRITE YOUR NAME HERE:
ישראל הבר
%%%%%%
}
\date{תאריך הגשה:
%%%%%% WRITE THE DATE OF SUBMISSION HERE:
יותר מדי קרוב
%%%%%%
}
\begin{document}
\maketitle

%%%%%%%% ERASE BEFORE SUBMISSION
ניתן
לכתוב נוסחאות בתוך השורה 
$f:G\rightarrow H$
וניתן לכתוב נוסחאות בשורה משלהן למשל
$$.f:\mathfrak g \rightarrow \mathfrak h$$
אפשר גם לכתוב שרשרת שיוויונות:
\begin{align*}
A & =B\\
 & =\frac{1}{2}C\\
 & \stackrel{\text{reason}}{\in}\mathbb F
\end{align*}
וגם אפשר לצייר מטריצות:
$$.\left[\begin{array}{cc}
a & \vec{v}\\
\vec{w} & b
\end{array}\right]$$
\selectlanguage{english}
Of course, you may also write in English but note the command used to change languages.
\selectlanguage{hebrew}
%%%%%% ERASE UNTIL HERE

\section*{תשובה לשאלה 1}

\subsection*{א.}
זה ממש סבבה לכתוב ככה כי הכל עובד והכל כיף והכל תותים וזה אפילו תראו - 
$y=x+17$
זה ממש פונקציה וזה כי אני ממש שמח וזה
\subsection*{ב.}


\subsection*{ג.}




\section*{תשובה לשאלה 2}

\subsection*{א.}


\subsection*{ב.}


\subsection*{ג.}


\subsection*{ד.}



\subsection*{ה.}

\section*{תשובה לשאלה 3}

\subsection*{א.}


\subsection*{ב.}






\end{document}

